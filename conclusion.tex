\section{CONCLUSION}\label{sec:conclusion}

\replaced{To make the architecture recovery process as complete and accurate as possible it might be essential the application of different analysis techniques. Previous studies have shown that the overall accuracy in the architecture recovery process using a recovery technique alone is still low.}{One of the factors that can influence the success or failure of a process of legacy system modernization is the understanding of its architecture. In this sense, the time taken to recover these concepts is as important as the time spent in planning the new system. This is due to the fact that, for a complete understanding of a legacy system, the first step is to understand its architecture, because this is the base that supports all system features. Thus, to make the architecture recovery process as complete and accurate as possible it might be essential the application of different analysis techniques. Previous studies have shown that the overall accuracy in the architecture recovery process using a recovery technique alone is still low.}

\deleted{In this context, the use of software visualization techniques for analysis and recovery of a system architecture is essential, since it allows greater flexibility to the process of modernization. Through this technique, it is possible to obtain a compact representation of the entire structure of the source code. The various aspects of the software implemented in several lines of code may be represented by a single diagram, which summarizes all this complexity.
In addition, an architecture recovery process using clustering techniques is also effective. A representation automatically created by a data collation process is a quick and convenient way to explore complex systems, often totally unknown to the analyst. This is is a good first step to understanding important aspects of the software. In other cases, the clustering process can provide different perspectives in understanding of software functionalities.}

To some extent, our work is the first of a kind where we explore the use of automatic and semi-automatic techniques, e.g. clustering and visualization, to jointly provide a more comprehensive as well as accurate architecture recovery. In this work, we make evidence for such claim in a industrial environment where our approach have significantly improved the architecture recovery process disregard the programming language under study (in this case Java and Visual Basic) or the order of the analysis techniques applied. We believe such representations will allow different views on various aspects of the software, which contributes to the understanding of the whole system structure. Thus, the use of these techniques can potentially allow for the recovery of a system architecture, with agility and accuracy. 

For future work, we plan to make a comprehensive analysis on public software repository, e.g. GitHub, and conduct a more thorough empirical study with the purpose of analysing not only other projects developed in other programming languages like C or C++ but also the scalability of our approach.