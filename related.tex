\section{RELATED WORK}\label{sec:related}

Maintenance and evolution of software systems is an expensive and challenging task. For this reason, several approaches for the recovery of software architecture have been proposed. Ducasse et al. in \cite{ducasse_software_2009} present a taxonomy of approaches to architecture recovery, detailing information necessary for recovery, such as: what are the stakeholders’ goals, how does the general reconstruction proceed, what are the available sources of information, based on this, which techniques can we apply, and, finally, what kind of knowledge does the process provide.

Garcia \textit{et al.} \cite{Garcia:ASE2013} performed a comparative analysis of six automated architecture recovery techniques. The selected techniques rely on two kinds of input obtained from im\-ple\-mentation-level artifacts: textual and structural. The accuracy of the techniques were asses on eight architectures from six different open-source systems. The results obtained indicate that two of the selected recovery techniques are superior to the rest along multiple measures. However, the results also show that there is significant room for improvement in all of the studied techniques.

Lutellier \textit{et al.} \cite{Lutellier_2015} compared nine variants of six architecture recovery techniques using two different types of dependencies: symbol and include. Four of the selected techniques use dependencies to determine clusters, while the remaining two techniques use textual information from source code. The results shows that symbol dependencies generally produce architectures with higher accuracies than include dependencies. Despite this improvement, the overall accuracy is low for all recovery techniques.

Regarding the use of semi-automatic techniques for understanding software systems, Wettel \textit{et al.} \cite{wettel_software_2011} carried out a controlled experiment to investigate the efficacy and effectiveness of the CodeCity software visualization tool in the process of understanding a software system structure. Results pointed out that, using such tool, it was possible to obtain an accuracy of 24.26\% and a reduction of 12.01\% in the execution time of a certain number of tasks, when compared to an understanding process carried out via a manual inspection of the source code.