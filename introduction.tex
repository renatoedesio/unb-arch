\section{Introduction}

\hyphenation{semi-au-to-mat-ic}

Software modernization is an activity that requires a significant understanding of the architecture. \added{Nevertheless, this is a challenge since there is usually a myriad of absent architecture information due to the lack of architecture specification documents, particularly w.r.t legacy systems.} As a result, some reverse engineering practices must take place to reconstruct the architectural structure of a software system. Moreover, the ability to mine relevant information from execution traces tend to quickly become large and unmanageable. For this reason, it is necessary to filter part of those traces and extract relevant information for the particular understanding of the task being performed \cite{canfora_achievements_2011}. As a result, the study of methods and tools to extract the software systems architecture is fundamental to guarantee a fast and accurate process. \deleted{In most cases, understanding the software structure is also a challenging task due to the lack of architecture specification documents, particularly w.r.t legacy systems. If there is no such specification available or if it is out-of-date, a software architecture recovery process must take place.}
%The required effort to manually recover a system architecture is proportional to the the size and complexity of its source code. 
%This is particular more costly for large software systems, prone to every sort of errors \cite{zapalowski_revealing_2014}. 
%As a result, the study of methods and tools to extract the software systems architecture is fundamental to guarantee a fast and accurate process. 
%By these means, costs and risks in the process of architecture recovery should be diminished.

Software visualization is a widely used technique for architecture recovering~\cite{ghanam_survey_2008, teyseyre_overview_2009, lanza_codecrawler-lessons_2003, Feijs_loe}. \added{They are \emph{semi-automatic} techniques that reconstruct the software architecture by manually abstracting low-level knowledge, 
due to interactive and expressive visualization tools \cite{ducasse_software_2009}.} The basis of the software visualization underlies on the creation of a representation of the system via visual elements~\cite{teyseyre_overview_2009}. Using such abstraction, it is possible to obtain a new view of the software, which allows exploring different concepts and clearer understanding of the software system structure.

Despite the improvement on the results obtained through \replaced{semi-automatic techniques}{software visualization techniques}, most of the times, 
the efficiency of the methods or tools rely on the ability of the analyst in charge of the software architecture recovery. As a result, the accuracy of the results,\added{particularly 
to derive the modular decomposition of the software}, is bound by subjective criteria most of the times. In order to fill in this gap, various research work has been devoted to automate the architecture recovery process. For instance, \added{software clusterization 
is an automatic technique} to recover a software system architecture\cite{shtern_clustering_2012, Jain_Murty_1999, Wiggerts, mitchell_heuristic_2002}. 
%\added{This technique assists software architects to identify groups of objects (such as a modular 
%decomposition of a system) whose members are similar in some way \cite{ducasse_software_2009}}.  
%Such technique is used in various fields 
%of research to describe methods to cluster unclassified data.  
In the context of architecture recovery, clustering techiniques aims at grouping software artifacts in significant modules, leading towards an understanding of 
the software system structure in a higher abstraction level \cite{shtern_clustering_2012}. However, Lutellier \textit{et al.} \cite{Lutellier_2015} has recently pointed out that, apart from the selection of the architecture recovery techniques, the accuracy was low for all studied techniques, corroborating past results \cite{Garcia:ASE2013}.

On the other hand, although both approaches for architecture recovery have been widely discussed in the literature, there is no empirical study that investigates how each technique complements each other with the purpose of achieving a more accurate architecture recovery process. In this paper we carried out an experiment that aims at understanding how the combination of both approaches improve the accuracy of the models obtained from an architectural recovery. The contributions of the paper are twofold:
\begin{itemize}
\item It presents the design, execution, and main findings of an empirical study that 
investigates the benefits of combining well-known semi-automatic visualization and clustering techniques for architecture recovering. 
\item Without loss of generality w.r.t the recovery process applied, it reports that the combination of the selected approaches can improve the accuracy of the resulting models \replaced{from 19\% to 30\%.}{ --- that improvement ranges from 19\% to 30\%, depending on the visualization and programming language techinique we used in our experiment.}   
\end{itemize}

% However, to decompose a software system in higher-level structures in a coherent and logic coupling, 
% is an inherently difficult task.  Due to the usual sparse set of data in a source code and the various 
% computational factors involved in a software system, many solutions are only suboptimal and do not represent 
% the real software architecture \cite {mitchell_heuristic_2002}.  On the other hand, results of such technique are 
% still helpful as they provide as a result a view of the software system decomposition, instead of reconstructing it from scratch. 
% Further refinements are then required in order to have an accurate architecture recovery \cite{tzerpos_comprehension-driven_2001}.

%Given the inaccuracy inherent to architecture recovery techniques, one question comes out:  Is it possible to obtain a more accurate software architecture recovery process? 
%We propose to investigate this question by merging both \replaced{quasi-manual and quasi-automatic techniques}{visualization and clustering techniques} into one  architecture 
%recovery process. We postulate that  \replaced{quasi-manual and quasi-automatic techniques, such as software visualization and clustering, respectively,} {clustering and visualization  techniques} 
%are complementary in the sense that they provide different levels of software architecture abstraction. Our investigation is performed on a controlled 
%experiment in a industrial environment, where participants are software analysts and developers working on software systems developed in Java and Visual Basic programming languages. 
%Our results show a statistically significant improvement in the accuracy of the models produced from our proposal compared to the techniques used alone.

The remaining sections of the paper are structured as follows: Section \ref{sec:related} presents related work regarding \replaced{semi-au\-to\-mat\-ic and automatic}{software architecture visualization and clustering} techniques. Section \ref{sec:rationale} highlights the inaccuracies of such techniques and the principles that drive our investigation. Section \ref{sec:experimental} is the core section where we present and evaluate the experimental study we conducted with the software development group in the Data Processing Center at University of Brasilia. In Section \ref{sec:conclusion} we conclude our work and present the future directions we plan to pursue.

